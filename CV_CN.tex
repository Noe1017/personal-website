%%%%%%%%%%%%%%%%%%%%%%%%%%%%%%%%%%%%%%%%%
% Medium Length Professional CV
% LaTeX Template
% Version 2.0 (8/5/13)
%
% This template has been downloaded from:
% http://www.LaTeXTemplates.com
%
% Original author:
% Trey Hunner (http://www.treyhunner.com/)
%
% Important note:
% This template requires the resume.cls file to be in the same directory as the
% .tex file. The resume.cls file provides the resume style used for structuring the
% document.
%
%%%%%%%%%%%%%%%%%%%%%%%%%%%%%%%%%%%%%%%%%

%----------------------------------------------------------------------------------------
%	PACKAGES AND OTHER DOCUMENT CONFIGURATIONS
%----------------------------------------------------------------------------------------
\documentclass{resume} % Use the custom resume.cls style
\usepackage[left=0.75in,top=0.6in,right=0.75in,bottom=0.6in]{geometry} % Document margins
\usepackage{xeCJK}
\usepackage{array}
\usepackage{color}
\usepackage[colorlinks=true,
            linkcolor=linkblue,
            urlcolor=linkblue,
            citecolor=linkblue]{hyperref}

\name{李吉祥} % Your name
\address{\href{mailto:jixiangli1017@gmail.com}{jixiangli1017@gmail.com} \\ \href{https://github.com/Noe1017}{github.com/Noe1017} \\ \href{https://jixiangli.xyz}{jixiangli.xyz}}
\begin{document}
\linespread{0.95}

%----------------------------------------------------------------------------------------
%	EDUCATION SECTION
%----------------------------------------------------------------------------------------

\begin{rSection}{教育背景}
\textbf{中北大学} \hfill 太原,中国\\
计算机科学与技术学士(工学) \hfill 2020年9月 - 2024年6月\\
(刘鼎工程荣誉班,前5\%)
\end{rSection}

%----------------------------------------------------------------------------------------
%	WORKING EXPERIENCE SECTION
%----------------------------------------------------------------------------------------

\begin{rSection}{工作经历}

\begin{rSubsection}{美团}{北京,中国}{\textup{大数据工程师}}{\textup{2024年2月 - 至今}}
\begin{itemize}
\setlength\itemsep{-0.5em}
\item 负责订单与交易数据域,专注于 ETL 流程、数据仓库优化、日常数据维护,使用 Flink、Doris 和 Kafka 支持实时数据处理,为营销、运营和策略场景提供大规模数据处理能力
\item 构建 AI 驱动的 Agent 平台,实现数据问答、SQL 指导和问题分解的自动化,减少人工支持并实现智能工作流编排
\item 主导数据资产管理,识别并消除历史数仓中的冗余数据,确保数据标准化并提升数据资产的可复用性
\item 优化低效的 Hadoop 和 Flink 任务,提升任务执行效率并缩短处理时间
\end{itemize}
\end{rSubsection}

\begin{rSubsection}{旷视科技}{北京,中国}{\textup{C++开发工程师(实习)}}{\textup{2023年6月 - 2023年10月}}
\begin{itemize}
\setlength\itemsep{-0.5em}
\item 负责图商通信与数据适配、EHP 数据校验及 PPL 模块的 C++ 开发,使用 KML 验证 SDMap 车道精度
\item 地图格式转换:实现了 OSM、Apollo 与私有标注格式之间的转换
\item 参与 SDMap 引擎开发并完成适配器接口封装
\end{itemize}
\end{rSubsection}

\end{rSection}

%----------------------------------------------------------------------------------------
%	PROJECTS / RESEARCH EXPERIENCE SECTION
%----------------------------------------------------------------------------------------

\begin{rSection}{黑客松}

\begin{rSubsection}{\href{https://github.com/Noe1017/SilentDAO}{\textup{Mantle Global Hackathon 2025}}}{线上}{\href{https://github.com/Noe1017/SilentDAO}{\textup{SilentDAO}}}{\textup{2025年12月}}
\begin{itemize}
\setlength\itemsep{-0.5em}
\item 设计并实现基于零知识证明(ZK)的匿名治理投票系统,在 Mantle L2 上实现可验证的 DAO 投票且不暴露投票者身份
\item 实现 ZK 成员证明与 nullifier 逻辑,在保护隐私的同时确保资格校验并防止重复投票
\end{itemize}
\end{rSubsection}

\begin{rSubsection}{\href{https://luma.com/o5rjsvpc?tk=qwiw43}{Monad Blitz}}{北京,中国}{\href{https://github.com/Noe1017/gas_morph}{\textup{gas\_morph}}}{\textup{2025年8月}}
\begin{itemize}
\setlength\itemsep{-0.5em}
\item 设计并实现基于 ERC-4337 Paymaster 的 Gas 补贴协议,实现意图驱动和基于权限的 Gas 赞助
\item 构建基于 NFT 的访问控制与会话管理,确保补贴策略的安全、可扩展及防滥用
\end{itemize}
\end{rSubsection}

\end{rSection}


%----------------------------------------------------------------------------------------
%	HONORS & AWARDS SECTION
%----------------------------------------------------------------------------------------

\begin{rSection}{荣誉奖项}
\begin{itemize}
\setlength\itemsep{-0.5em}
\item \href{https://jixiangli.xyz/awards/2024-ICPC-China-National-Invitational-Contest-Jixiang-Li-MEDAL.pdf}{银奖,ACM-ICPC 中国全国邀请赛},2023年6月
\item \href{https://jixiangli.xyz/awards/CCCC-TPD-Ladder-Contest.pdf}{全国一等奖,CCCC —— TPD 团体程序设计天梯赛},2023年5月
\item \href{https://jixiangli.xyz/awards/2023-Asia-Xian-Regional-Contest-Jixiang-Li-MEDAL.pdf}{铜奖,ACM-ICPC 亚洲区域赛},2022年11月
\item \href{https://jixiangli.xyz/awards/Lanqiao-Cup.jpg}{全国二等奖,蓝桥杯},2022年6月
\end{itemize}
\end{rSection}

%----------------------------------------------------------------------------------------
%	OTHERS SECTION
%----------------------------------------------------------------------------------------

\begin{rSection}{兴趣爱好}
热衷于参与各类 Web3 黑客松,探索 AI 编程技术。对区块链开发、去中心化应用以及利用 AI 工具进行软件开发充满热情。
\end{rSection}

\end{document}
